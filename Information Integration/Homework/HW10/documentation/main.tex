\documentclass{article}

\usepackage{fancyhdr}
\usepackage{extramarks}
\usepackage{amsmath}
\usepackage{amsthm}
\usepackage{amsfonts}
\usepackage{tikz}
\usepackage{algpseudocode}
\usepackage{caption}
\usepackage{subcaption}
\usepackage{listings}
\usepackage{color}


\usetikzlibrary{automata,positioning}

%
% Basic Document Settings
%

% Custom sectioning
%\usepackage{sectsty}
%\allsectionsfont{\centering \normalfont\scshape}

\captionsetup{justification=centering,
singlelinecheck=false
}
\graphicspath{ {required/} }

\topmargin=-0.45in
\evensidemargin=0in
\oddsidemargin=0in
\textwidth=6.5in
\textheight=9.0in
\headsep=0.25in

\linespread{1.1}

\pagestyle{fancy}
\lhead{\hmwkAuthorName}
\chead{\hmwkClass}
\rhead{\hmwkTitle}
\lfoot{\lastxmark}
\cfoot{\thepage}

\renewcommand\headrulewidth{0.4pt}
\renewcommand\footrulewidth{0.4pt}

\setlength\parindent{0pt}

%
% Create Problem Sections
%

\newcommand{\enterProblemHeader}[1]{
    \nobreak\extramarks{}{Problem \arabic{#1} continued on next page\ldots}\nobreak{}
    \nobreak\extramarks{Problem \arabic{#1} (continued)}{Problem \arabic{#1} continued on next page\ldots}\nobreak{}
}

\newcommand{\exitProblemHeader}[1]{
    \nobreak\extramarks{Problem \arabic{#1} (continued)}{Problem \arabic{#1} continued on next page\ldots}\nobreak{}
    \stepcounter{#1}
    \nobreak\extramarks{Problem \arabic{#1}}{}\nobreak{}
}

\setcounter{secnumdepth}{0}
\newcounter{partCounter}
\newcounter{homeworkProblemCounter}
\setcounter{homeworkProblemCounter}{1}
\nobreak\extramarks{Problem \arabic{homeworkProblemCounter}}{}\nobreak{}

%
% Homework Problem Environment
%
% This environment takes an optional argument. When given, it will adjust the
% problem counter. This is useful for when the problems given for your
% assignment aren't sequential. See the last 3 problems of this template for an
% example.
%
\newenvironment{homeworkProblem}[1][-1]{
    \ifnum#1>0
        \setcounter{homeworkProblemCounter}{#1}
    \fi
    \section{Problem \arabic{homeworkProblemCounter}}
    \setcounter{partCounter}{1}
    \enterProblemHeader{homeworkProblemCounter}
}{
    \exitProblemHeader{homeworkProblemCounter}
}

%
% Homework Details
%   - Title
%   - Due date
%   - Class
%   - Section/Time
%   - Instructor
%   - Author
%

\newcommand{\hmwkTitle}{Homework\ 10}
\newcommand{\hmwkDueDate}{April 3, 2015}
\newcommand{\hmwkClass}{Information Integration on the Web}
\newcommand{\hmwkClassInstructor}{Prof. Ambite \& Knoblock}
\newcommand{\hmwkAuthorName}{Tushar Tiwari}

%
% Title Page
%

\title{
    \vspace{2in}
    \textmd{\textbf{\hmwkClass:\ \hmwkTitle}}\\
    \normalsize\vspace{0.1in}\small{Due\ on\ \hmwkDueDate}\\
    \vspace{0.1in}\large{\textit{\hmwkClassInstructor}}
    \vspace{3in}
}

\author{\textbf{\hmwkAuthorName}}
\date{}

\renewcommand{\part}[1]{\textbf{\large Part \Alph{partCounter}}\stepcounter{partCounter}\\}

%
% Various Helper Commands
%

% Useful for algorithms
\newcommand{\alg}[1]{\textsc{\bfseries \footnotesize #1}}

% For derivatives
\newcommand{\deriv}[1]{\frac{\mathrm{d}}{\mathrm{d}x} (#1)}

% For partial derivatives
\newcommand{\pderiv}[2]{\frac{\partial}{\partial #1} (#2)}

% Integral dx
\newcommand{\dx}{\mathrm{d}x}

% Alias for the Solution section header
\newcommand{\solution}{\section{\textbf{\Large Solution}}}

% Probability commands: Expectation, Variance, Covariance, Bias
\newcommand{\E}{\mathrm{E}}
\newcommand{\Var}{\mathrm{Var}}
\newcommand{\Cov}{\mathrm{Cov}}
\newcommand{\Bias}{\mathrm{Bias}}

% Language Definitions for SPARQL
\definecolor{olivegreen}{rgb}{0.2,0.8,0.5}
\definecolor{grey}{rgb}{0.5,0.5,0.5}
\lstdefinelanguage{ttl}{
sensitive=true,
morecomment=[l][\color{grey}]{PREFIX},
morecomment=[l][\color{grey}]{BASE},
morecomment=[l][\color{olivegreen}]{SELECT},
morestring=[b][\color{blue}]",
showstringspaces=false,
}
\lstset{%
  numbers=none,
  tabsize=4,
  breaklines=true,
  basicstyle=\small\ttfamily,
  framerule=10pt,
  columns=fullflexible,
  language=ttl
}

\begin{document}

\maketitle

\pagebreak
% Language Definitions for Turtle
\definecolor{olivegreen}{rgb}{0.2,0.8,0.5}
\definecolor{grey}{rgb}{0.5,0.5,0.5}
\lstdefinelanguage{ttl}{
sensitive=true,
morecomment=[l][\color{grey}]{@},
morecomment=[l][\color{grey}]{PREFIX},
morecomment=[l][\color{grey}]{BASE},
morecomment=[l][\color{olivegreen}]{\#},
morestring=[b][\color{blue}]\",
showstringspaces=false,
}
\lstset{%
  numbers=none,
  tabsize=4,
  breaklines=true,
  basicstyle=\small\ttfamily,
  framerule=10pt,
  columns=fullflexible,
  language=ttl
}

\begin{homeworkProblem}
    Describe the performed record linkage
    \solution
    \subsubsection{Data Transformation}
    After loading both the files, a few transformations were performed in order to get better matches.
    \begin{itemize}
    \item Transform name, addr and type to lower cased strings. This ensures that different casings of the field do not cause a wrong match. For example, "steakhouse" and "Steakhouse" may not give a perfect edit distance score even though they are spelled the same way.
    \item Trimmed any leading and trailing whitespaces
    \item Transformed phone num of d1.csv from "xxx/xxx-xxxx" to "xxx-xxx-xxxx" using regex.
    \end{itemize}
    \subsubsection{Metrics for matching}
    \begin{itemize}
    \item In the given data I immediately noticed that a lot of the matches have the exact same phone number assosciated with them. That is a matching pair of records shared the same phone number. This is however not true for all, but for most. Hence, I added a q grams comparision metric on phone no with weight 60\% because if they have the exact same phone number then they must be a matching pair. The q value was chosen to be 4 because most restaurants will share the same area code so it must be atleast more than 3.
    \item After using the above metric, I noticed that the some of the records that did not match from the above metric had same names but different phone nums and hence I added an equal fields boolean distance metric on the name column. Obviously records with exactly the same name will be a matching pair. The weightage I gave this was 30\% because the soundex metric explained below added another 10\%.
    \item A phone match was causing lots of matches with similar confidence levels and to put one match ahead of the other match, I added a metric that performed a soundex on the name with weightage 10\%. If there were two matches on phone numbers, then the one with more similar soundex name would get selected.
    \item I chose an acceptance level of 40\% because records with same name should get accepted if their phones do not match. 30\% from exact and 10\% from soundex.
    \end{itemize}
    \subsubsection{F-Score}
    \begin{itemize}
    \item The output file generated (output.csv) by fril contains 112 total matches.
    \item The groundtruth file also contains 112 matches.
    \item The number of true positive matches is 109.
    \item $precision = 109 / 112 = 0.97321427 $
    \item $recall = 109 / 112 = 0.97321427$
    \item $F-Score = 2 \cdot \frac{precision \cdot recall}{precision + recall} = 0.97321427$
    \end{itemize}
    
\end{homeworkProblem}
\end{document}